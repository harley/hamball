\documentclass{amsart}
\usepackage{pslatex}
\usepackage{ifthen}
\usepackage{latexsym}
\usepackage{theorem}
\usepackage{graphicx}
\usepackage{epsfig}
\usepackage{amssymb}
\usepackage{amsmath}
\usepackage{amsfonts}
\usepackage{txfonts}
\usepackage{url}
\usepackage{verbatim}
\usepackage{fullpage}
\usepackage{moreverb}
\usepackage{color}
\title{Hamster Balls}
\author{David Costanzo}
\author{Alexander Thomson}
\author{Harley Trung}
\author{Matt Sills}

\begin{document}
\pagestyle{empty}
\maketitle

\section*{Introduction}
\noindent
Hamster Balls is a multiplayer first person shooter implemented entirely in Haskell and
built on the Functional Reactive Programming framework Yampa. To the best of
our (considerably extensive) knowledge, Hamster Balls is the first 3D network game ever
implemented in Haskell. 

\section*{Basic Game Mechanics}
\noindent
Each player controls a hamster armed with a 10-terawatt laser and a
100-terawatt protective hamster ball. Mouse-look and standard \texttt{W-A-S-D} controls are used
to aim and move. The game's objective is to shoot and destroy other players. Every laser
hit a player takes reduces that player's protective hamster ball strength by the wattage
of the laser fired, and when a laser hit takes a hamster's protection down to 0 strength,
the hamster dies a gory death and is respawned at its original start location. Once
a player achieves a kill, his name is displayed in the top-left corner of the screen along
with his score.

\section*{Advanced Computer Graphics Features}
\noindent
We include two advanced features in our current implementation of Hamster Balls:
a Mandelbrot-generated waterfall comprised of 1058 triangles flows (well, hangs)
from the top of a crudely-rendered Watson Hall and a lightning-fast N$^2$ time collision
detection algorithm bounces uncoordinated players off walls.

\section*{Future Work}
\noindent
Over the course of the summer we intend to improve the game further. Our ultimate goal is
to improve it to the point where it is played competitively in the zoo by bored undergrads,
then release it as an open source project. Features currently under development: 360$^{\circ}$
vision, powerups, faster and more robust collision detection, additional maps, team play,
variable mouse sensitivity, gravity, sound effects, and cuter hamsters.

\section*{Running the Game}
\noindent
Run \url{/home/accts/agt5/hamsters/binServer} from any zoo machine to start up the server. Then run
\url{/home/accts/agt5/hamsters/binClient} \url{<playerName>} to join the game. Note that the game is full screen, so 
only one instance can be played per computer. To fully appreciate the game, you will want to 
have at least two players logged in. Note that once any player exits the game (by pressing \texttt{Esc}), there
is a very high probability that the server and other clients will crash in the near future. Also, make
sure you restart the server with each run. We anticipate that these issues will be resolved soon.

\vspace{10pt} \noindent
We also have a dynamically-spawnable waterfall in our game. It's generated by computing values of the Mandelbrot
set, so it's a fairly complex structure. Unfortunately, this also means that it takes much longer to render than
do other shapes. Thus we allow the user to choose to spawn or despawn it. Press \texttt{L} to spawn and \texttt{K}
to despawn the Mandelfall/Waterbrot.

\end{document}


